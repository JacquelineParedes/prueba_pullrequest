\documentclass[11pt,letterpaper]{article}
\usepackage[utf8x]{inputenc}
\usepackage{graphicx}

\title{Resumen del Propedeutico 2018}
\author{Alumnos del grupo LCC1}
\begin{document}
\maketitle
\section{Terminal}
\subsection{¿Por qué usar la terminar?}
%equipo1		
\subsection{Comandos basicos}
%equipo2		
\subsection{Permisos}
%equipo3
\section{Emacs}
\subsection{¿Qué es y por qué usar emacs?}
%equipo4
\subsection{Comandos básicos}
%equipo5
\section{Git}
\subsection{¿Qué es git y por qué usarlo?}
%equipo6
\subsection{Comandos básicos}
%equipo7
\subsection{GitHub}
%equipo8		
\subsection{Inicialiar un resposotorio remoto}
%equipo9		
\subsection{Colaborar en repositorio}
%equipo10		
\subsection{Pull Request}
%equipo11
\section{Lenguajes de Marcado}
\subsection{¿Qué son para que se usan?}
%equipo12		
\subsection{XML}
%equipo13
\subsection{¿Qué es HTML y para que se usa?}
%equipo14		
es\subsection{Etiquetas Básicas}

%equipo15
\section{Latex}
\subsection{¿Qué es Latex y por qué usarlo?}
es un sistema de composición de textos, orientado a la creación de documentos escritos que presenten una alta calidad tipográfica. Por sus características y posibilidades, es usado de forma especialmente intensa en la generación de artículos y libros científicos que incluyen, entre otros elementos, expresiones matemáticas.


Funciona y es estable y multiplataforma.

    Tan simple como eso, LaTeX no se cuelga, el formato de los archivos es mucho más estable que en otros procesadores y cualquier cambio es primero profundamente meditado y después profusamente documentado, existen implementaciones para distintas plataformas y en todas el resultado es exactamente el mismo (si se tienen los mismos estilos y tipos, claro).
Alta calidad en la edición de ecuaciones.

    Esta es siempre la razón última por la que un usuario científico se inclina hacia LaTeX. Este procesador ajusta los tamaños de paréntesis, integrales, subíndices y superíndices, alinea los elementos de las matrices, construye cajas, etc.
LaTeX permite redactar fácilmente documentos estructurados.

    A través de distintas clases de documento y de su conjunto de macros, LaTeX posibilita escribir textos dividiéndolos en capítulos, secciones, subsecciones, controlando en todo momento la numeración y las referencias cruzadas. Construye índices de contenidos, tablas o figuras. Ajusta los tamaños y tipos de letras según la parte del documento en que se hallen.
Facilidad en la construcción de macros y órdenes.

    A poco de comenzar a usar este procesador, el usuario se encuentra definiendo o redefiniendo órdenes para que éstas se ajusten a sus preferencias personales. Por ejemplo, es posible que una determinada expresión aparezca repetidas veces en el texto. Nada mas fácil que definir una orden que reemplace a todo un bloque. O bien, es posible que no guste la forma en que LaTeX numera las páginas. Una redefinición al principio del documento permite cambiar esto.
Se escribe en ASCII.

    Esto, que al principio puede parecer un inconveniente (ya que implica teclear mucho más) se torna en ventaja al cabo del tiempo. Por un lado permite incrementar la velocidad de escritura (pues no hay que andar utilizando ratón o menús), por otro facilita el uso de cualquier editor de texto (no contiene caracteres de control) y permite su transmisión por correo electrónico (puede escribirse en ASCII de 7 bits). Esto hace que muchas revistas científicas admitan artículos escritos en LaTeX, enviados por e-mail. Ellos lo procesan en el lugar de destino, hacen los cambios necesarios y lo imprimen.
Es gratis

    Pues eso. Para ver porqué véase la sección ¿Porqué TeX es gratis?. A pesar de ello, existen también implementaciones comerciales.

%equipo16
\subsection{Estructura básica de un documento en Latex}rigin
%equipo17
\end{document}
