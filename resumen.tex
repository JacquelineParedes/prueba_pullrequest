\documentclass[11pt,letterpaper]{article}
\usepackage[utf8x]{inputenc}
\usepackage{graphicx}

\title{Resumen del Propedeutico 2018}
\author{Alumnos del grupo LCC1}
\begin{document}
\maketitle
\section{Terminal}
\subsection{¿Por qué usar la terminar?}


%equipo1
\subsection{Comandos basicos}
\begin{itemize}
\item cd
  \begin {itemize}
  \item Change directory - Para cambiar el directorio
    \end{itemize}
\item ls
  \begin{itemize}
  \item List - Muestra todo lo que esta dentro del directorio donde estas posisionado
  \end{itemize}
\item mkdir
  \begin{itemize}
  \item Make directory - Crear un directorio
  \end{itemize}
\item rm
  \begin{itemize}
  \item Remove - Quitar un archivo o directorio
  \end{itemize}
\end{itemize}



%equipo2
  \subsection{Permisos}
  \begin{itemize}
    
  \item r
    \begin{itemize}
    \item Read - Permiso para leer un archivo y ver que hay adentro de un directorio (Y permiso para remover un directorio)
    \end {itemize}
    
  \item w
    \begin {itemize}
    \item Write - Permiso para escribir dentro de un archivo y modificar lo que hay adentro de un directorio
    \end {itemize}
    
  \item x
    \begin {itemize}
    \item Execute - Permiso para ejecutar un archivo y abri un directorio
    \end {itemize}
  \end {itemize}

  
%equipo3
\section{Emacs}
\subsection{¿Qué es y por qué usar emacs?}
%equipo4
\subsection{Comandos básicos}
%equipo5
\section{Git}
\subsection{¿Qué es git y por qué usarlo?}
%equipo6
\subsection{Comandos básicos}
%equipo7
\subsection{GitHub}
%equipo8		
\subsection{Inicialiar un resposotorio remoto}
%equipo9		
\subsection{Colaborar en repositorio}
%equipo10		
\subsection{Pull Request}
%equipo11
\section{Lenguajes de Marcado}
\subsection{¿Qué son para que se usan?}
%equipo12		
\subsection{XML}
%equipo13
\subsection{¿Qué es HTML y para que se usa?}
%equipo14		
\subsection{Etiquetas Básicas}
%equipo15
\section{Latex}
\subsection{¿Qué es Latex y por qué usarlo?}
%equipo16
\subsection{Estructura básica de un documento en Latex}
%equipo17
\end{document}
