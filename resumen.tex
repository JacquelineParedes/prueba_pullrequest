\documentclass[11pt,letterpaper]{article}
\usepackage[utf8x]{inputenc}
\usepackage{graphicx}

\title{Resumen del Propedeutico 2018}
\author{Alumnos del grupo LCC1}
\begin{document}
\maketitle
\section{Terminal}
\subsection{¿Por qué usar la terminar?}
%equipo1		
\subsection{Comandos basicos}
%equipo2		
\subsection{Permisos}
%equipo3
\section{Emacs}
\subsection{¿Qué es y por qué usar emacs?}
%equipo4
\subsection{Comandos básicos}
%equipo5
\section{Git}
\subsection{¿Qué es git y por qué usarlo?}
%equipo6
\subsection{Comandos básicos}
%equipo7
\subsection{GitHub}
%equipo8		
\subsection{Inicialiar un resposotorio remoto}
%equipo9		
\subsection{Colaborar en repositorio}
%equipo10		
\subsection{Pull Request}
%equipo11
\section{Lenguajes de Marcado}
\subsection{¿Qué son para que se usan?}
%equipo12		
\subsection{XML}
Es un lengueje de marcado que se creo para el metalenguaje SGML, en donde se puede organizar la informacion jerarquicamente.
En este se puede ir especificando la estructura de un documento y para esto se iran añadiendo una a una las partes que lo integraran.
%equipo13
\subsection{¿Qué es HTML y para que se usa?}
%equipo14		
\subsection{Etiquetas Básicas}
%equipo15
\section{Latex}
\subsection{¿Qué es Latex y por qué usarlo?}
%equipo16
\subsection{Estructura básica de un documento en Latex}
%equipo17
\end{document}
